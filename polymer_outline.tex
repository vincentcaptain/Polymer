\documentclass[11pt, oneside]{article}   	% use "amsart" instead of "article" for AMSLaTeX format
\usepackage{geometry}                		% See geometry.pdf to learn the layout options. There are lots.
\geometry{letterpaper}                   		% ... or a4paper or a5paper or ... 
%\geometry{landscape}                		% Activate for rotated page geometry
%\usepackage[parfill]{parskip}    		% Activate to begin paragraphs with an empty line rather than an indent
\usepackage{graphicx}				% Use pdf, png, jpg, or eps§ with pdflatex; use eps in DVI mode
								% TeX will automatically convert eps --> pdf in pdflatex		
\usepackage{amssymb}

%SetFonts

%SetFonts
\begin{document}
\section{Reproduce the result from paper using LAMMPS}
\par 
Step 1 requires (in LJ units):  �
\par
Interaction between molecules ($\epsilon$) mimicked by harmonic oscillator vanished off at 3$\sigma$. NVE with langevin. Thermostat without thinking about the on/off states and force. 
\par

\par Parameters:
N=10,� R=2,� L=2,� dt=0.01 in thermostat,� $r_0$=0.5 for the bond length,� $s_s$
=2 for spring constant (bonds)	$\epsilon$=2.5, $\epsilon_w$=0.3 for the LJ constant between monomers and between wall particles, $f_{hr}$=5 within region 1 and only acting on the first monomer, $f_{pull}$=1.0 is acting on the first monomer iff in region 2, $200^3$ 3D box, ignore $k_{on} = 10$ and $k_{off} = 1$ for now.
\par Problems:
\par Lost atom (solved by discretizing the time in more detail). (Solved)
\par "Walls" are generated by repulsion, original method can cause diffusion. (Solved)
\par The positions and momentum of all atoms should be defined in data file. (Solved)
\par Force design: parameter $x_{lo}$ and $x_{hi}$ as the range of action, $\Delta$ as scaling, $f_{max}$ as maximum force added, $\omega$ as frequency and t as time. $F = \frac{tanh(\frac{x-x_{lo}}{\Delta})-tanh(\frac{x-x_{hi}}{\Delta})}{2}f_{max}cos(\omega t)$
\par FFS for different $\omega$: initial flux estimated by diffusion constant 0.00245 via curve fitting. 
\par step 2: cut the LJ pairing between polymers and wall atoms at 1.12 instead of 3 to avoid overlapping around the corner; calculating the free energy change from the wall to anywhere left by $-k_BTln(P)$ where P is subtracted from the spring we fixes by umbrella sampling.
\section{Revise the model into 3D} (solved)
\section{Try models in different attraction forces besides harmonic approximation}
\par Hydrogen bonding, fene and electrostatics
\section{Heterogeneous monomers}
\section{Design initial structures for peptides}


































\end{document}  